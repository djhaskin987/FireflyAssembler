\documentclass[final,hyperref={pdfpagelabels=false}]{beamer}
\usepackage{grffile}
\mode<presentation>{\usetheme{byu}}
\usepackage[english]{babel}
\usepackage[latin1]{inputenc}
\usepackage{amsmath,amsthm, amssymb, latexsym}
\boldmath
\usepackage[orientation=landscape,size=a0,scale=1.5]{beamerposter}
\usepackage{array,booktabs,tabularx}
\newcolumntype{Z}{>{\centering\arraybackslash}X}
\listfiles
\graphicspath{{figures/}}
\title{\huge Firefly Assembler: Parallelizing the Assembly of Genome Fragments}
\author{Kyle Corbitt, Dan Haskin, Perry Ridge}
\institute[Brigham Young University]{Molecular Biology and Computer Science Departments}
\date[Nov. 20th, 2013]{Nov. 20th, 2013}
\newlength{\columnheight}
\setlength{\columnheight}{105cm}
\begin{document}
\begin{frame}
  \begin{columns}
    \begin{column}{.49\textwidth}
      \begin{beamercolorbox}[center,wd=\textwidth]{postercolumn}
        \begin{minipage}[T]{.95\textwidth}
          \parbox[t][\columnheight]{\textwidth}{
            \begin{block}{Purpose}
                Our purpose is to create a genome assembler based on a novel
                approach called the Firefly Algorithm (1).  We hypothesize that
                this will produce more accurate assemblies than other assemblers.
            \end{block}
            \begin{block}{Introduction}
                Currently the computational time required to accurately
                assemble a de novo genome is a significant bottleneck in
                bioinformatics.  Algorithms exist that can assemble a genome in
                a relatively short time, but these algorithms typically are not
                able to guarantee an accurate assembly. Conversely, algorithms
                exist that can guarantee an accurate assembly of reads but that
                are unacceptably inefficient.  By modeling genome assembly as a
                Travelling Salesman Problem (TSP), we show how established
                algorithms for getting near-optimal results on that problem
                can be used for the purpose of solving the genome assembly
                problem.
            \end{block}
            \begin{block}{Feature Extraction}
                I am a doof.
            \end{block}
          }
        \end{minipage}
      \end{beamercolorbox}
    \end{column}
    \begin{column}{.49\textwidth}
      \begin{beamercolorbox}[center,wd=\textwidth]{postercolumn}
        \begin{minipage}[T]{.95\textwidth}
          \parbox[t][\columnheight]{\textwidth}{
            \begin{block}{Introduction}
                Something way cool here.
            \end{block}
            \vfill
            \begin{block}{Feature Extraction}
                I am a doof.
            \end{block}
          }
        \end{minipage}
      \end{beamercolorbox}
    \end{column}
  \end{columns}
  \vskip1ex
  \tiny\hfill{Created with \LaTeX \texttt{beamerposter}  \url{http://www-i6.informatik.rwth-aachen.de/~dreuw/latexbeamerposter.php} \hskip1em}
\end{frame}
\end{document}
